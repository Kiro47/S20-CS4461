\documentclass[12pt,addpoints,answers]{exam}
\usepackage[utf8]{inputenc}

\unframedsolutions
\renewcommand{\solutiontitle}{\noindent\textbf{Solution:}\par\noindent}
\SolutionEmphasis{\color{blue}}
\printanswers

\usepackage{booktabs}
\usepackage{tabularx}
\usepackage{url}
\usepackage{xfrac}

\usepackage{siunitx}
\sisetup{parse-numbers=false}

\usepackage{listings}
\lstset{basicstyle=\scriptsize\ttfamily}

\usepackage{tikz}
\usetikzlibrary{arrows}
\usetikzlibrary{decorations.pathreplacing}
\usetikzlibrary{chains}
\usetikzlibrary{positioning}
\usetikzlibrary{shapes.geometric}
\usetikzlibrary{shapes.multipart, calc, decorations.pathreplacing}
%\tikzset{>=stealth',every on chain/.append style={join}, every join/.style={-,blue,thick,dashed}}


\title{Computer Networks Homework 2}
\author{Fall 2019}
\date{Due: 24 February 2020}

\begin{document}
\maketitle

\begin{questions}
\question[6] List at least three responsibilities of the Transport Layer.
\begin{solution}
	The Transport Layer is responsible for the following:
	\begin{enumerate}
		\item Creating an end to end \textit{connection} between two hosts.
		\item Controlling the flow of packets in a way that does not stress the two connected devices.
		\item Ensuring a complete and verified data transfer in TCP based connections.
	\end{enumerate}
\end{solution}

\question[5] A sequence of UDP packets addressed to ports 1979, 8854, 9000, 1979, and 9000 arrive at a host (in that order). Draw a simple diagram that shows how these packets enter the demux service and then how they are distributed to the processess.
\begin{solution}
	\tikzstyle{startstop} = [rectangle, rounded corners, minimum width=3cm, minimum height=1cm,text centered, draw=black, fill=red!30]
	\tikzstyle{io} = [trapezium, trapezium left angle=70, trapezium right angle=110, minimum width=2cm, minimum height=0.5cm, text centered, draw=black, fill=blue!30]
	\tikzstyle{process} = [rectangle, minimum width=2cm, minimum height=0.5cm, text centered, draw=black, fill=orange!30]
	\tikzstyle{decision} = [diamond, minimum width=2cm, minimum height=0.5cm, text centered, draw=black, fill=green!30]
	\tikzstyle{arrow} = [thick,->,>=stealth]
	
	% Begin Diagram
\begin{tikzpicture}[node distance=2cm]
	\node (start) [startstop, xshift=2cm] {Incoming connection from another host};
	
	% Start nodes
	\node (multiplex) [process, below of=start, xshift=2cm] {Transport Layer Multiplexer};
	\draw [arrow] (start) -- node[anchor=west, xshift=1cm] {[Packets: 1979, 8854, 9000, 1979, 9000]} (multiplex);
	
	% Multiplex
	\node (proto) [io, below of=multiplex, xshift=2cm] {Input};
	\draw [arrow] (multiplex) -- node[anchor=west, xshift=1cm] {[Packets: 1979, 8854, 9000, 1979, 9000]} (proto);
	
	% Protocols
	\node (udp) [decision, below of=proto, xshift=-2cm] {UDP};
	\draw [arrow] (proto) -- node[anchor=east,yshift=0.15cm] {UDP Protocol [Packets: 1979, 8854, 9000, 1979, 9000]} (udp);
	\node (tcp) [decision, right of=udp, xshift=2cm] {TCP};
	\draw [arrow] (proto) -- node[anchor=west,yshift=0.15cm] {TCP Protocol}(tcp);
	
	\node (tcpEnd) [io, below of=tcp] {Ports};
	\draw [arrow] (tcp) -- node[anchor=west,xshift=0.1cm] {No packets come here}(tcpEnd);
	
	% UDP snapshot
	\node (udpPorts) [io, below of=udp] {Ports};
	\draw [arrow] (udp) -- (udpPorts);
	
	% Port Split
	\node (udp1979) [decision, below of=udpPorts, xshift=-4cm] {1979};
	\draw [arrow] (udpPorts) -- node[anchor=east,yshift=0.15cm] {Consume: 1979}(udp1979);
	
	\node (udp8854) [decision, below of=udpPorts] {8854};
	\draw [arrow] (udpPorts) -- (udp8854);
	
	\node (udp9000) [decision, below of=udpPorts, xshift=4cm] {9000};
	\draw [arrow] (udpPorts) -- (udp9000);
	
%%% Split 1
	\draw (-5,-11) -- (13,-11);
	
	% UDP snapshot
	\node (udpPorts1) [io, below of=udp8854] {Ports};
	\node (udpPorts1Text) [right of=udpPorts1, anchor=west] {[Packets: 8854, 9000, 1979, 9000]};
	
	% Port Split
	\node (udp1979a) [decision, below of=udpPorts1, xshift=-4cm] {1979};
	\draw [arrow] (udpPorts1) -- (udp1979a);
	
	\node (udp8854a) [decision, below of=udpPorts1] {8854};
	\draw [arrow] (udpPorts1) -- node[anchor=east,yshift=-0.15cm] {Consume: 8854}(udp8854a);
	
	\node (udp9000a) [decision, below of=udpPorts1, xshift=4cm] {9000};
	\draw [arrow] (udpPorts1) -- (udp9000a);

	\end{tikzpicture}
	
	\begin{tikzpicture}[node distance=2cm]
	
	%%% Split 2
	\draw (-5,1) -- (13,1);
	
	% UDP snapshot
	\node (udpPorts2) [io] {Ports};
	\node (udpPorts2Text) [right of=udpPorts2, anchor=west] {[Packets: 9000, 1979, 9000]};
	
	% Port Split 
	\node (udp1979b) [decision, below of=udpPorts2, xshift=-4cm] {1979};
	\draw [arrow] (udpPorts2) -- (udp1979b);
	
	\node (udp8854b) [decision, below of=udpPorts2] {8854};
	\draw [arrow] (udpPorts2) -- (udp8854b);
	
	\node (udp9000b) [decision, below of=udpPorts2, xshift=4cm] {9000};
	\draw [arrow] (udpPorts2) -- node[anchor=west,yshift=0.15cm] {Consume: 9000}(udp9000b);
	
	%%% Split 3
		\draw (-5,-3) -- (13,-3);
	
	% UDP snapshot
	\node (udpPorts3) [io, below of=udp8854b] {Ports};
	\node (udpPorts3Text) [right of=udpPorts3, anchor=west] {[Packets: 1979, 9000]};
	
	
	% Port Split
	\node (udp1979c) [decision, below of=udpPorts3, xshift=-4cm] {1979};
	\draw [arrow] (udpPorts3) -- node[anchor=east,yshift=0.15cm] {Consume: 1979}(udp1979c);
	
	\node (udp8854c) [decision, below of=udpPorts3] {8854};
	\draw [arrow] (udpPorts3) -- (udp8854c);
	
	\node (udp9000c) [decision, below of=udpPorts3, xshift=4cm] {9000};
	\draw [arrow] (udpPorts3) -- (udp9000c);
	
	
	
	%%% Split 4
		\draw (-5,-7.1) -- (13,-7.1);
	
	% UDP snapshot
	\node (udpPorts4) [io, below of=udp8854c] {Ports};
	\node (udpPorts4Text) [right of=udpPorts4, anchor=west] {[Packets: 9000]};
	
	
	% Port Split
	\node (udp1979d) [decision, below of=udpPorts4, xshift=-4cm] {1979};
	\draw [arrow] (udpPorts4) -- (udp1979d);
	
	\node (udp8854d) [decision, below of=udpPorts4] {8854};
	\draw [arrow] (udpPorts4) -- (udp8854d);
	
	\node (udp9000d) [decision, below of=udpPorts4, xshift=4cm] {9000};
	\draw [arrow] (udpPorts4) -- node[anchor=west,yshift=0.15cm] {Consume: 9000}(udp9000d);
	
	% Programs
		\draw (-5,-12) -- (13,-12);
	\node (aggregatorText) [below of=udp8854d, yshift=-1cm] {All port consumptions directly map this way};
	
	\node (udp1979e) [decision, below of=aggregatorText, xshift=-4cm] {1979};
	\node (programA) [process, below of=udp1979e] {Program A};
	\draw [arrow] (udp1979e) -- node[anchor=west,yshift=0.15cm] {Packet x 2} (programA) ;

	\node (udp8854e) [decision, below of=aggregatorText] {8854};
	\node (programB) [process, below of=udp8854e] {Program B};
	\draw [arrow] (udp8854e) -- node[anchor=west,yshift=0.15cm] {Packet x 1} (programB);

	\node (udp9000e) [decision, below of=aggregatorText, xshift=4cm] {9000};
	\node (programC) [process, below of=udp9000e] {Program C};
	\draw [arrow] (udp9000e) -- node[anchor=west,yshift=0.15cm] {Packet x 2} (programC);
	\end{tikzpicture}
	

\end{solution}



\question[9] Use RFC1700 to find the well-known port numbers for the following services: SMTP, HTTP, Kerberos, POP3, BGP, WHOAMI, LDAP, and RPC (remote process execution). Be careful, RPC has multiple port assignments; list at least two and the associated application or algorithm.
\begin{solution}
	\begin{itemize}
		\item SMTP:
			\begin{itemize}
				\item 25
				\item 465
				\item 587
				\item 3535 (semi-common but unofficial)
			\end{itemize}
		\item HTTP: 
			\begin{itemize}
				\item 80
				\item 443 (It's not exactly HTTP, but they're almost always paired)
				\item 8080 (Mostly used for testing, but very common)
				\\
				\\
				Many other services use HTTP as their program layer, but will not be listed (as there's hundreds)
			\end{itemize}

		\item Kerberos:
			\begin{itemize}
				\item 543 (Login)
				\item 544 (Remote Shell)
				\item 749 (Administration)
				\item 750 (Version IV)
				\item 751 (Master Authentication)
				\item 752 (Password Server)
				\item 753 (User Reg)
				\item 754 (Slave Propagator)
				\item 760 (Registration) 
			\end{itemize}
		\item POP3:
			\begin{itemize}
				\item 110
				\item 995 (SSL/TLS)
			\end{itemize}
		\item BGP:
			\begin{itemize}
				\item 179
			\end{itemize}
		\item WHOAMI:
			\begin{itemize}
				\item 565
				\item https://www.ietf.org/rfc/rfc1340.txt
			\end{itemize}
		\item LDAP:
			\begin{itemize}
				\item 389
				\item 636
				\item 3268 (Microsoft Active Directory LDAP)
			\end{itemize}
		\item RPC:
			\begin{itemize}
				\item 530 (generic)
				\item 593 (HTTP RPC) (EX: Microsoft Mail Exchange)
				\item 8069 (OpenERP - XML RPC) (EX: Odoo)
			\end{itemize}
	\end{itemize}
\end{solution}

\question[16] A UDP segment has the payload ``MTU!'' as a string of four ASCII charaacters. It is being delivered to an application on port 5272 from an application on port 8888. Show the full UDP header for this segment.

Clearly mark each field with it's purpose and represent the value of each field in binary. Use an \SI{8}{b} representation for the ASCII characters (a convenient ASCII table is available at \url{www.asciitable.com}). Show all 1s Complement arthmetic for the checksum calculation, remembering that the checksum is based on 16-bit chunks. As a ``real'' checksum would require information from the IP packet you should just use the information from UDP and ignore the pseudoheader when calculating your checksum.
\begin{solution}
	UDP Datagram Header:
	\begin{itemize}
		\item Source Port: 0001010010011000
		\item Destination Port: 0010001010111000
		\item Length: 0000000000001100
		\item Checksum: 1101100111010001
	\end{itemize}
\end{solution}

\question[4] Find at least two other important UDP applications besides DHCP, DNS, and QUIC. Provide a citation which proves the application uses UDP.
\begin{solution}
	\begin{itemize}
		\item TFTP:  https://tools.ietf.org/html/rfc1350
		\item NFSv2: https://tools.ietf.org/html/rfc1094
	\end{itemize}
\end{solution}

\question[8] List the two components of any protocol that adheres to the precepts of Automatic Repeat Requests. How do these two components work together to provide reliable delivery of packets?
\begin{solution}
\end{solution}

\question[2] What problem is solved by adding a single-bit sequence number to packets using the Stop-and-Wait protocol?
\begin{solution}
\end{solution}

\question[5] You are designing a protocol that uses the sliding window protocol. You have targeted a window size of 16 frames. What is the minimum number of bits needed in the header field for packets of this protocol?
\begin{solution}
\end{solution}

\question[3] TCP uses a \SI{32}{b} sequence number. What is the maximum window size that could be used?
\begin{solution}
\end{solution}

\question[12] Show the states the TCP state machine moves through in the following situations:
\begin{parts}
\part The client initiates a connection, then changes its mind with a \lstinline{close()} function.
\part A standard closing of the connection from the \lstinline{Established} state by the client.
\part The client initiates a new connection, but the server never responds.
\end{parts}
\begin{solution}
\end{solution}

\question[4] They may sound similar, but explain the differences between the \lstinline{PUSH} and the \lstinline{URGENT} flags in a TCP segment.
\begin{solution}
\end{solution}

\question[6] Several segments are sent by a client containing \SI{852}{b}, \SI{777}{b}, and \SI{212}{b}. Prior to these transmissions, the last ACK received by the client was for sequence number \texttt{0x001109AB}. What are the sequence numbers and ACKs for these segments?
\begin{solution}
\end{solution}

\question[10] A TCP host is using the original timeout equations and 10 packets arrive with the following RTT values: \SI{145}{\milli\second}, \SI{145}{\milli\second}, \SI{155}{\milli\second}, \SI{144}{\milli\second}, \SI{100}{\milli\second}, \SI{255}{\milli\second}, \SI{200}{\milli\second}, \SI{185}{\milli\second}, \SI{166}{\milli\second}, \SI{145}{\milli\second}. Show how the value of the value of $\mathrm{EstRTT}$ changes as each packet arrives. Start with a \SI{100}{\milli\second} value for $\mathrm{EstRTT}$.
\begin{solution}
\end{solution}

\question[10] Repeat the previous question using the Jacobson/Karels equations. Let the important values be: $\mathrm{EstimatedRTT} = \SI{100}{\milli\second}$, $\mathrm{Deviation} = 0$, $\mu = 1$, $\gamma = 4$, $\delta = 0.875$.

\end{questions}
\end{document}
